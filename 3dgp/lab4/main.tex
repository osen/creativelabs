\documentclass[10pt]{article}

\usepackage[a4paper, total={6in, 10in}]{geometry}
\usepackage{fancyvrb}
\usepackage{color}
\usepackage{xcolor}
\usepackage{framed}
\usepackage{environ}
\setlength{\parindent}{0pt}

%\usepackage{fancyvrb}
\usepackage{xcolor}
\usepackage{float}
\usepackage{graphicx}
\usepackage{listings}
\usepackage{mdframed}
\usepackage{tcolorbox}

\definecolor{customnavy}{RGB}{0,0,100}

\lstnewenvironment{CodeInner}{%
}{%
}

\newenvironment{CodeOld}{%
\begin{CodeInner}}{%
\end{CodeInner}}

%\newenvironment{Code}
%{\VerbatimEnvironment
%\begin{mdframed}[backgroundcolor=customnavy,fontcolor=white,linewidth=0]
%\begin{Verbatim}
%}{\end{Verbatim}\end{mdframed}}

\newenvironment{Code}
{\VerbatimEnvironment
\begin{tcolorbox}[arc=0pt,outer arc=0pt,before skip=10px, after skip=10px, colframe=white, colback=customnavy, fontupper=\color{white}]
\begin{Verbatim}
}{\end{Verbatim}\end{tcolorbox}}

\newcommand{\Todo}[1]{\begin{center}\colorbox{red!30}{\parbox{0.9\textwidth}{\textbf{\textit{TODO}}\\#1}}\end{center}}
\newcommand{\Note}[1]{\begin{center}\colorbox{gray!30}{\parbox{1.0\textwidth}{\textbf{\textit{Note:}}\\#1}}\end{center}}

\newcommand{\Name}[1]{\textit{#1}}
\newcommand{\Command}[1]{\textbf{\textit{#1}}}

\newcommand{\Ref}[1]{\textbf{Figure \ref{fig:#1}}}
\newcommand{\Sec}[1]{\textbf{Section \ref{sec:#1}}}

\newcommand{\Diagram}[1]{\begin{figure}[H]
  \centering
  \includegraphics[width=0.5\textwidth]{#1}
  \end{figure}}

\newcommand{\Figure}[3]{\begin{figure}[H]
  \centering
  \includegraphics[width=0.5\textwidth]{#1}
  \caption{\textit{#2}}
  \label{fig:#3}
  \end{figure}}

\newcommand{\FigureFull}[3]{\begin{figure}[H]
  \centering
  \includegraphics[width=1.0\textwidth]{#1}
  \caption{\textit{#2}}
  \label{fig:#3}
  \end{figure}}


\begin{document}

\title{3D Graphics Programming \\
  \large Lab 5: Matrix}

\author{Karsten Pedersen\\ Department of Creative Technology}
\maketitle

In this lab we will be looking into making your project look as though
it is moving.  In OpenGL, we do not modify the data stored in the VBOs
because doing so would be very expensive each frame. Instead we keep
the data as it is and instead perform calculations on the positions as
we draw the shape.  We do this in the Vertex shader.

The first step is to modify the shader and add a new uniform in the
same way we did when changing the color. The only difference is that
instead of a color (vec4), we want to pass in a mat4 which can contain
the transform.  This will be the model matrix. We also want to pass in
another matrix called the Projection matrix so that we can add perspective
into our project. This will help allow us to see the triangle move unlike
if we remain in normalised device coordinates.

\begin{Code}

uniform mat4 in_Projection;
uniform mat4 in_Model;

\end{Code}

We now multiply the \Name{in\_Position} by the new model matrix and projection
matrix.  This transforms the vertex coordinates into world space and
then into screen space respectively.

\begin{Code}

attribute vec3 in_Position;
attribute vec4 in_Color;

varying vec4 ex_Color;

void main()
{
  gl_Position = in_Projection * in_Model * vec4(in_Position, 1.0);
  ex_Color = in_Color;
}

\end{Code}

Now in the C++ we need to generate our perspective projection matrix
and assign it to the \Name{in\_Projection} variable.  Likewise we also need to
create our model matrix and assign it to \Name{in\_Model}. Finally we draw the shape.

\begin{Code}

float angle = 0;

// Inside the main loop

// Draw with perspective projection matrix
shader->setUniform("in_Projection", glm::perspective(glm::radians(45.0f),
 (float)WINDOW_WIDTH / (float)WINDOW_HEIGHT, 0.1f, 100.f));

glm::mat4 model(1.0f);
model = glm::translate(model, glm::vec3(0, 0, -2.5f));
model = glm::rotate(model, glm::radians(angle), glm::vec3(0, 1, 0));

shader->setUniform("in_Model", model);
shader->draw(shape);

angle += 0.1f;

\end{Code}

\Note{
In this lab we are using the wrapper code to cut down on the amount
of code we need to write to perform this task.  The main thing to
note that the setUniform function uses \Name{glGetUniformLocation}
to obtain the uniform location and \Name{glUniformMatrix4fv} to assign
the matrix. The exact function call is glUniformMatrix4fv(uniformId, 1,
GL\_FALSE, glm::value\_ptr(value)); \Name{uniformId} is the \Name{GLint}
containing the uniform variable location and \Name{value} is the matrix
to assign. The \Name{glm::value\_ptr} is to extract the raw matrix out
of the glm wrapper for use with OpenGL.
}

With that working, you should be able to increase \Name{angle} per frame
and see the triangle spin. Now try to also draw the same triangle but with
an orthographic projection matrix.  The following code demonstrates this:

\begin{Code}

// Draw with orthographic projection matrix
model = glm::mat4(1.0f);

shader->setUniform("in_Projection", glm::ortho(0.0f,
  (float)WINDOW_WIDTH, 0.0f, (float)WINDOW_HEIGHT, 0.0f, 1.0f));

model = glm::translate(model, glm::vec3(100, WINDOW_HEIGHT - 100, 0));
model = glm::scale(model, glm::vec3(100, 100, 1));

shader->setUniform("in_Model", model);
shader->draw(shape);

\end{Code}

By the end of this lab, your project should look similar to that in \Ref{current}.

\Figure{images/triangle}{The final output of the rotating triangle}{current}

\end{document}
