\documentclass[10pt]{article}

\usepackage[a4paper, total={6in, 10in}]{geometry}
\usepackage{fancyvrb}
\usepackage{color}
\usepackage{xcolor}
\usepackage{framed}
\usepackage{environ}
\setlength{\parindent}{0pt}

%\usepackage{fancyvrb}
\usepackage{xcolor}
\usepackage{float}
\usepackage{graphicx}
\usepackage{listings}
\usepackage{mdframed}
\usepackage{tcolorbox}

\definecolor{customnavy}{RGB}{0,0,100}

\lstnewenvironment{CodeInner}{%
}{%
}

\newenvironment{CodeOld}{%
\begin{CodeInner}}{%
\end{CodeInner}}

%\newenvironment{Code}
%{\VerbatimEnvironment
%\begin{mdframed}[backgroundcolor=customnavy,fontcolor=white,linewidth=0]
%\begin{Verbatim}
%}{\end{Verbatim}\end{mdframed}}

\newenvironment{Code}
{\VerbatimEnvironment
\begin{tcolorbox}[arc=0pt,outer arc=0pt,before skip=10px, after skip=10px, colframe=white, colback=customnavy, fontupper=\color{white}]
\begin{Verbatim}
}{\end{Verbatim}\end{tcolorbox}}

\newcommand{\Todo}[1]{\begin{center}\colorbox{red!30}{\parbox{0.9\textwidth}{\textbf{\textit{TODO}}\\#1}}\end{center}}
\newcommand{\Note}[1]{\begin{center}\colorbox{gray!30}{\parbox{1.0\textwidth}{\textbf{\textit{Note:}}\\#1}}\end{center}}

\newcommand{\Name}[1]{\textit{#1}}
\newcommand{\Command}[1]{\textbf{\textit{#1}}}

\newcommand{\Ref}[1]{\textbf{Figure \ref{fig:#1}}}
\newcommand{\Sec}[1]{\textbf{Section \ref{sec:#1}}}

\newcommand{\Diagram}[1]{\begin{figure}[H]
  \centering
  \includegraphics[width=0.5\textwidth]{#1}
  \end{figure}}

\newcommand{\Figure}[3]{\begin{figure}[H]
  \centering
  \includegraphics[width=0.5\textwidth]{#1}
  \caption{\textit{#2}}
  \label{fig:#3}
  \end{figure}}

\newcommand{\FigureFull}[3]{\begin{figure}[H]
  \centering
  \includegraphics[width=1.0\textwidth]{#1}
  \caption{\textit{#2}}
  \label{fig:#3}
  \end{figure}}


\begin{document}

\title{3D Graphics Programming \\
  \large Lab 2: Experimenting with Shaders}

\author{Karsten Pedersen\\ Department of Creative Technology}
\maketitle

Now that we have a triangle rendering we can start experimenting with
the shader code. We currently are displaying a blue triangle. We know
that if we change it in the fragment shader to be a different colour,
we can make it any colour we need. However, using this method it would
be hard to draw a number of differently coloured triangles. We would
need a different shader for each one! This is where \Name{Uniforms} come in.

\subsection*{Uniforms}

You can think of a \Name{Uniform} as a variable within the shader that you
can change from your \Name{C/C++} code.  In order use a \Name{Uniform}
we need to modify the respective shader object. Since we want to set
the color via our \Name{Uniform}, we will be editing the \Name{Fragment
Shader}.

\begin{Code}
const GLchar *fragmentShaderSrc =
  "uniform vec4 in_Color;" \
  "void main()" \
  "{" \
  "  gl_FragColor = in_Color;" \
  "}" \
  "";
\end{Code}

We add a new \Command{vec4} variable (RGBA) to hold the color and
we label it as \Command{uniform} (in a similar way to how we did the
triangle position attribute).  We then use the color directly rather
than the hard coded blue value.  If you try to run the program now you
will probably see a black triangle. This is because the uniform is at
the default value of \Command{(0, 0, 0, 0)} (Black). So what we now need to do
is set the uniform value.

\

The first step is to find the uniform from the shader (\textbf{After}
the shader has been linked via \Command{glLinkProgram}, unlike when you
were binding the attribute location).

\begin{Code}
// Store location of the color uniform and check if sucessfully found
GLint colorUniformId = glGetUniformLocation(programId, "in_Color");

if(colorUniformId == -1)
{
  throw std::exception();
}
\end{Code}

\Note{
You will notice that we are using a \Command{GLint} rather than a
\Command{GLuint}. This means it is a signed integer and the main reason
for this is that unlike with the shader, the value of \Command{0} is
valid (i.e the first uniform) and instead \Command{-1} is given on an
error. Only signed integral types are able to hold a negative value
correctly.
}

With all this in place, at any point within the program we can change the color
used by the shader. The following code demonstrates this and you should end up with
output similar to \Ref{final}.

\begin{Code}
// Bind the shader to change the uniform, set the uniform and reset state
glUseProgram(programId);
glUniform4f(colorUniformId, 0, 1, 0, 1);
glUseProgram(0);
\end{Code}

\Figure{images/final}{A screenshot of our triangle with its color set via a \Name{Uniform}}{final}

\Note{
Unlike obtaining the uniform location via \Command{glGetUniformLocation},
the \Command{glUniform4f} function does not take an argument of the
shader program to operate on. This is a bit messy because instead
you need to bind and unbind the current shader to operate on via
\Command{glUseProgram}. This state machine driven approach is one of
the main complaints developers have with \Name{OpenGL} over something
like i.e \Name{DirectX}. In practice it is just something you have to
get used to and make sure via diciplined coding you keep resetting the
state back to a known state. Once you have written helper functions or
even a game engine on top of \Name{OpenGL}, it will become much easier.
}

\subsection*{Attributes}

If we wanted each point of the triangle to have a different color,
light value or texture coordinate (such as in \Ref{final_attribute}), a
\Name{Uniform} would not be viable because that remains constant during
the drawing of the entire shape rather than per vertex. Instead we use
the exact same technique as we did for the positions of the triangle in
the first place using \Name{Attribute streams}.

\Figure{images/final_attribute}{A screenshot of our triangle with its color set via an \Name{Attribute Stream}}{final_attribute}

One slight complexity with \Name{Attributes} is that they always need to be passed in to the \Name{Vertex shader}
and passed through into the \Name{Fragment shader} so that they can be interpolated between the two. For this reason we
will need to replace the \Name{Vertex shader} with the following:

\begin{Code}
const GLchar *vertexShaderSrc =
  "attribute vec3 in_Position;" \
  "attribute vec4 in_Color;" \
  "" \
  "varying vec4 ex_Color;" \
  "" \
  "void main()" \
  "{" \
  "  gl_Position = vec4(in_Position, 1.0);" \
  "  ex_Color = in_Color;" \
  "}" \
  "";
\end{Code}

Here you will see that we have an additional \Name{Attribute}
called \Command{in\_Color} that we pass to another variable of type
\Command{varying} via the simple assignment in the \Command{main}
function. This is all that is required to pass a variable through from
the \Name{Vertex shader} to the \Name{Fragment shader}. Next we will simply use this
value from within the \Name{Fragment shader} as the final output color. The following
listing shows this simple process.

\begin{Code}
const GLchar *fragmentShaderSrc =
  "varying vec4 ex_Color;" \
  "void main()" \
  "{" \
  "  gl_FragColor = ex_Color;" \
  "}" \
  "";
\end{Code}

\Note{
The variable name \Command{ex\_Color} \textbf{must} be the same in
each shader, otherwise \Name{OpenGL} does not know how the two variables
link together.
}

Next, as we did with out positions \Name{Attribute stream} we define
the colors of each point of the triangle (red, green, blue).

\begin{Code}
const GLfloat colors[] = {
  1.0f, 0.0f, 0.0f, 1.0f,
  0.0f, 1.0f, 0.0f, 1.0f,
  0.0f, 0.0f, 1.0f, 1.0f
};
\end{Code}

We then do the process of creating the new \Name{VBO}, uploading the
data onto it and assigning it to position \Command{1} of the \Name{VAO}.

\begin{Code}
GLuint colorsVboId = 0;

// Create a colors VBO on the GPU and bind it
glGenBuffers(1, &colorsVboId);

if(!colorsVboId)
{
  throw std::exception();
}

glBindBuffer(GL_ARRAY_BUFFER, colorsVboId);

// Upload a copy of the data from memory into the new VBO
glBufferData(GL_ARRAY_BUFFER, sizeof(colors), colors, GL_STATIC_DRAW);

// Bind the color VBO, assign it to position 1 on the bound VAO and flag it to be used
glBindBuffer(GL_ARRAY_BUFFER, colorsVboId);
glVertexAttribPointer(1, 4, GL_FLOAT, GL_FALSE, 4 * sizeof(GLfloat), (void *)0);
glEnableVertexAttribArray(1)
\end{Code}

\Note{
Take extra notice of the seemingly magic numbers. We use \Command{1}
rather than \Command{0} because our color buffer is the second position
of the \Name{VAO}. We also use \Command{4} instead of \Command{3} because
our RGBA color buffer is a \Command{vec4} rather than a \Command{vec3}.
}

Before linking your \Name{Shader program}, remember to instruct
\Name{OpenGL} that you want to connect the \Name{in\_Color}
\Name{Attribute} variable with the second \Name{Attribute stream}
(the colors) within the \Name{VAO}.

\begin{Code}
glBindAttribLocation(programId, 1, "in_Color");
\end{Code}

You should now have a colorful triangle appearing when you run the
program. If you find that you are getting different results then very
carefully go through the code and check it. There is a lot of similar
code here and it is very easy to make a mistake.  We will look at a
slightly more scalable architecture in the next lab.

\Note{
So far we have covered using \Command{glUniform4f} to assign a
\Command{vec4} to the shader program.  Try using others such as
\Command{glUniform1f} in order to assign a \Command{uniform float}.
See if you can assign a \Name{uniform} in the \Name{Vertex shader}
in order to get the triangle to move left or right.
}

\end{document}
