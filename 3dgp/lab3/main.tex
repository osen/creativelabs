\documentclass[10pt]{article}

\usepackage[a4paper, total={6in, 10in}]{geometry}
\usepackage{fancyvrb}
\usepackage{color}
\usepackage{xcolor}
\usepackage{framed}
\usepackage{environ}
\setlength{\parindent}{0pt}

%\usepackage{fancyvrb}
\usepackage{xcolor}
\usepackage{float}
\usepackage{graphicx}
\usepackage{listings}
\usepackage{mdframed}
\usepackage{tcolorbox}

\definecolor{customnavy}{RGB}{0,0,100}

\lstnewenvironment{CodeInner}{%
}{%
}

\newenvironment{CodeOld}{%
\begin{CodeInner}}{%
\end{CodeInner}}

%\newenvironment{Code}
%{\VerbatimEnvironment
%\begin{mdframed}[backgroundcolor=customnavy,fontcolor=white,linewidth=0]
%\begin{Verbatim}
%}{\end{Verbatim}\end{mdframed}}

\newenvironment{Code}
{\VerbatimEnvironment
\begin{tcolorbox}[arc=0pt,outer arc=0pt,before skip=10px, after skip=10px, colframe=white, colback=customnavy, fontupper=\color{white}]
\begin{Verbatim}
}{\end{Verbatim}\end{tcolorbox}}

\newcommand{\Todo}[1]{\begin{center}\colorbox{red!30}{\parbox{0.9\textwidth}{\textbf{\textit{TODO}}\\#1}}\end{center}}
\newcommand{\Note}[1]{\begin{center}\colorbox{gray!30}{\parbox{1.0\textwidth}{\textbf{\textit{Note:}}\\#1}}\end{center}}

\newcommand{\Name}[1]{\textit{#1}}
\newcommand{\Command}[1]{\textbf{\textit{#1}}}

\newcommand{\Ref}[1]{\textbf{Figure \ref{fig:#1}}}
\newcommand{\Sec}[1]{\textbf{Section \ref{sec:#1}}}

\newcommand{\Diagram}[1]{\begin{figure}[H]
  \centering
  \includegraphics[width=0.5\textwidth]{#1}
  \end{figure}}

\newcommand{\Figure}[3]{\begin{figure}[H]
  \centering
  \includegraphics[width=0.5\textwidth]{#1}
  \caption{\textit{#2}}
  \label{fig:#3}
  \end{figure}}

\newcommand{\FigureFull}[3]{\begin{figure}[H]
  \centering
  \includegraphics[width=1.0\textwidth]{#1}
  \caption{\textit{#2}}
  \label{fig:#3}
  \end{figure}}


\begin{document}

\title{3D Graphics Programming \\
  \large Lab 4: 3D Program Architecture}

\author{Karsten Pedersen\\ Department of Creative Technology}
\maketitle

At this point, your code may be starting to get a little bit too large
to manage effectively (around 200 lines just to render a triangle!). In
this lab we are going to focus on separating it out into multiple classes
and make a small drawing API that uses \Name{OpenGL} underneath. You
can almost think of it as your own ``mini game engine''.  There are many
ways you could design your software so lets first cover the main tasks
that your code already performs.

\begin{enumerate}
  \item Create position vertex buffer
  \item Create colour vertex buffer
  \item Create vertex array consisting of both buffers
  \item Load a shader
  \item \textbf{[optional]} Assign specific colour to the shader using a uniform
  \item Draw the vertex array using the loaded shader
\end{enumerate}

With these steps in mind, take a look at your code that you have been
developing in the past few labs. Try to match it up to the individual
steps (via comments). These are the same parts that we are going to
separate out into small and reusable classes.

\

Now take a look at the following code as an example.

\begin{Code}
  VertexBuffer *positions = new VertexBuffer();
  positions->add(glm::vec3(0.0f, 0.5f, 0.0f));
  positions->add(glm::vec3(-0.5f, -0.5f, 0.0f));
  positions->add(glm::vec3(0.5f, -0.5f, 0.0f));

  VertexBuffer *colors = new VertexBuffer();
  colors->add(glm::vec4(1.0f, 0.0f, 0.0f, 1.0f));
  colors->add(glm::vec4(0.0f, 1.0f, 0.0f, 1.0f));
  colors->add(glm::vec4(0.0f, 0.0f, 1.0f, 1.0f));

  VertexArray *shape = new VertexArray();
  shape->setBuffer("in_Position", positions);
  shape->setBuffer("in_Color", colors);

  ShaderProgram *shader = new ShaderProgram("simple.vert", "simple.frag");

  // A little bit later on...

  shader->setUniform("in_Lightness", 0.5f);
  shader->draw(shape);
\end{Code}

The code should be fairly self documenting and you should see how it
can also match up against the exact same steps as yours. Whilst this
code is noticeably shorter and easier to read than your current code,
it largely does the same tasks. The complex parts are simply hidden
away within the respective classes. Something similar to this will be
the end goal when organising your own code. Try to attempt this now.

\Note{
Spend a good deal of time on this task, perhaps even outside of this
lab. Having a convenient and easy to work with framework is key for
making good progress with \Name{OpenGL}. You will be especially glad
you did as the coursework assignment nears!

}

The API demonstrated by the code listing above comes from a simplified
version of a library I hacked together so I can easily fiddle about with
\Name{OpenGL} to try out new techniques. \Ref{uml} shows a UML diagram of
it in its entirety so you can see it is pretty small. We will certainly
be expanding our code throughout this unit.


\FigureFull{images/uml}{A diagram showing a potential architecture wrapping the \Name{OpenGL} core concepts}{uml}

\Note{
The task of wrapping \Name{OpenGL} into your own classes is a great way
to consolidate your understanding of the API.  It will be fairly tricky
so I suggest you do a single class at a time rather than ripping it all up
and starting from a fresh canvas.

}

\Figure{images/current}{Because it felt weird not having a triangle on the lab sheet ;)}{current}

Once you have your API working. Try to draw another triangle. You should now find it much easier than before.
If that is working well, then perhaps attempt a square!

\end{document}
