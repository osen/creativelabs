\documentclass[10pt]{article}

\usepackage[a4paper, total={6in, 8in}]{geometry}
\usepackage{fancyvrb}
\usepackage{color}
\usepackage{xcolor}
\usepackage{framed}
\usepackage{environ}
\setlength{\parindent}{0pt}

%\usepackage{fancyvrb}
\usepackage{xcolor}
\usepackage{float}
\usepackage{graphicx}
\usepackage{listings}
\usepackage{mdframed}
\usepackage{tcolorbox}

\definecolor{customnavy}{RGB}{0,0,100}

\lstnewenvironment{CodeInner}{%
}{%
}

\newenvironment{CodeOld}{%
\begin{CodeInner}}{%
\end{CodeInner}}

%\newenvironment{Code}
%{\VerbatimEnvironment
%\begin{mdframed}[backgroundcolor=customnavy,fontcolor=white,linewidth=0]
%\begin{Verbatim}
%}{\end{Verbatim}\end{mdframed}}

\newenvironment{Code}
{\VerbatimEnvironment
\begin{tcolorbox}[arc=0pt,outer arc=0pt,before skip=10px, after skip=10px, colframe=white, colback=customnavy, fontupper=\color{white}]
\begin{Verbatim}
}{\end{Verbatim}\end{tcolorbox}}

\newcommand{\Todo}[1]{\begin{center}\colorbox{red!30}{\parbox{0.9\textwidth}{\textbf{\textit{TODO}}\\#1}}\end{center}}
\newcommand{\Note}[1]{\begin{center}\colorbox{gray!30}{\parbox{1.0\textwidth}{\textbf{\textit{Note:}}\\#1}}\end{center}}

\newcommand{\Name}[1]{\textit{#1}}
\newcommand{\Command}[1]{\textbf{\textit{#1}}}

\newcommand{\Ref}[1]{\textbf{Figure \ref{fig:#1}}}
\newcommand{\Sec}[1]{\textbf{Section \ref{sec:#1}}}

\newcommand{\Diagram}[1]{\begin{figure}[H]
  \centering
  \includegraphics[width=0.5\textwidth]{#1}
  \end{figure}}

\newcommand{\Figure}[3]{\begin{figure}[H]
  \centering
  \includegraphics[width=0.5\textwidth]{#1}
  \caption{\textit{#2}}
  \label{fig:#3}
  \end{figure}}

\newcommand{\FigureFull}[3]{\begin{figure}[H]
  \centering
  \includegraphics[width=1.0\textwidth]{#1}
  \caption{\textit{#2}}
  \label{fig:#3}
  \end{figure}}


\begin{document}

\title{3D Graphics Programming \\
  \normalsize Lab 1: Rendering a Triangle}

\author{Karsten Pedersen\\ Department of Creative Technology}
\maketitle

In this unit we aim to use an industry standard graphics API to draw
and interact with 3D objects. \Name{OpenGL} as an API is extremely flexible
but also comes with quite a bit of complexity.  Developing with \Name{OpenGL}
can often require some initial time to get used to how it works and also
to set up the data used for the rendering.

\

The first thing we will need to do is to ensure that the project is
set up and ready to use \Name{OpenGL}. To just get a window appearing
on the screen we will be using \Name{SDL 2} because not only is the code
required to do so fairly trivial but also because you are already familiar
with it.  Other libraries that are also common for this task include
\Name{[Free]Glut}, \Name{GLFW}, \Name{SFML} and \Name{Allegro}. Open
the provided \Name{Microsoft Visual Studio} project or generate
the \Name{CMake} build system.  The code inside \Command{main.cpp}
should be familiar to you by now. It simply opens up the \Name{SDL 2}
Window and not much else.  You will see that we are not even creating an
\Command{SDL\_Renderer} because we are going to be using \Name{OpenGL}
directly rather than using the basic renderer provided by \Name{SDL 2}
(which uses either \Name{OpenGL} or \Name{DirectX} underneath depending
on platform.

\Note{
\Name{Microsoft Windows} provides a version of \Name{OpenGL} which
is extremely old (version \textbf{1.2}) and is too inflexible to make
what we see today as modern games. Luckily a much newer implementation
of \Name{OpenGL} is provided by the graphics card manufacturer's driver
(i.e \Name{NVIDIA}, \Name{AMD}, \Name{Intel}, \Name{MESA}). As of writing
we are at around version \textbf{4.6} which only the very latest hardware
supports.  However the techniques and functionality covered in these labs
was actually provided by version \textbf{2.1} which means that your code
will work on almost all hardware found in the wild.

\

For convenience, rather than use a specific header and library
(i.e \Command{GL/gl.h}) from each different vendor's SDK, we
instead use a 3rd party library called \Name{Glew} (Open\textbf{GL}
\textbf{E}xtension \textbf{W}rangler). This library links the vendor's
specific implementation with the platforms implementation at runtime so
we can just use \Name{OpenGL} as usual and not worry about the details.

}

First, lets begin by including the \Name{SDL 2}/\Name{OpenGL} binding
layer header and the \Name{Glew} header file into our project.

\begin{Listing}

  #include <SDL2/SDL_opengl.h>
  #include <GL/glew.h>

\end{Listing}

Next we need to create an \Name{OpenGL} rendering context within
the created \Name{SDL 2} window and because \Name{Glew} loads the
\Name{OpenGL} library and extensions at runtime we also need to initialize
it. In your project, just after where you open the \Name{SDL 2} window
using \Command{SDL\_CreateWindow}, add the following function calls.

\begin{Listing}

  if(!SDL_GL_CreateContext(window))
  {
    throw std::exception();
  }

  if(glewInit() != GLEW_OK)
  {
    throw std::exception();
  }

\end{Listing}

\Note{
This must be \textbf{after} the call to open the window because otherwise
there is no window for \Name{OpenGL} to bind a context to and the call
will fail.
}

With this in place we are now ready to start with \Name{OpenGL}. What
we are first going to do to confirm everything is working is change
the screen to the color red. The following listing will first set the
current \Name{OpenGL} clear color to red, will then actually instruct
\Name{OpenGL} to clear the screen and finally it will atomically swap
the \Name{OpenGL} memory buffer with that of the screen buffer (to
eliminate flicker).

\begin{Listing}

  glClearColor(1.0f, 0.0f, 0.0f, 1.0f);
  glClear(GL_COLOR_BUFFER_BIT);
  SDL_GL_SwapWindow(window);

\end{Listing}

With that in place, compile the project and run it. You should hopefully
see the following.

\Todo{Big red OpenGL window}

Now we can finally begin using the main essence of \Name{OpenGL}. In 3D
graphics almost everything is made up of triangles so lets start with
trying to draw one. \Ref{process} should give you an overview
of the tasks required to do this.

\Figure{process}{Diagram showing the series of tasks in order to render a triangle}{process}

\end{document}
