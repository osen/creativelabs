\documentclass[10pt]{article}

\usepackage[a4paper, total={6in, 10in}]{geometry}
\usepackage{fancyvrb}
\usepackage{color}
\usepackage{xcolor}
\usepackage{framed}
\usepackage{environ}
\usepackage{hyperref}
\setlength{\parindent}{0pt}

%\usepackage{fancyvrb}
\usepackage{xcolor}
\usepackage{float}
\usepackage{graphicx}
\usepackage{listings}
\usepackage{mdframed}
\usepackage{tcolorbox}

\definecolor{customnavy}{RGB}{0,0,100}

\lstnewenvironment{CodeInner}{%
}{%
}

\newenvironment{CodeOld}{%
\begin{CodeInner}}{%
\end{CodeInner}}

%\newenvironment{Code}
%{\VerbatimEnvironment
%\begin{mdframed}[backgroundcolor=customnavy,fontcolor=white,linewidth=0]
%\begin{Verbatim}
%}{\end{Verbatim}\end{mdframed}}

\newenvironment{Code}
{\VerbatimEnvironment
\begin{tcolorbox}[arc=0pt,outer arc=0pt,before skip=10px, after skip=10px, colframe=white, colback=customnavy, fontupper=\color{white}]
\begin{Verbatim}
}{\end{Verbatim}\end{tcolorbox}}

\newcommand{\Todo}[1]{\begin{center}\colorbox{red!30}{\parbox{0.9\textwidth}{\textbf{\textit{TODO}}\\#1}}\end{center}}
\newcommand{\Note}[1]{\begin{center}\colorbox{gray!30}{\parbox{1.0\textwidth}{\textbf{\textit{Note:}}\\#1}}\end{center}}

\newcommand{\Name}[1]{\textit{#1}}
\newcommand{\Command}[1]{\textbf{\textit{#1}}}

\newcommand{\Ref}[1]{\textbf{Figure \ref{fig:#1}}}
\newcommand{\Sec}[1]{\textbf{Section \ref{sec:#1}}}

\newcommand{\Diagram}[1]{\begin{figure}[H]
  \centering
  \includegraphics[width=0.5\textwidth]{#1}
  \end{figure}}

\newcommand{\Figure}[3]{\begin{figure}[H]
  \centering
  \includegraphics[width=0.5\textwidth]{#1}
  \caption{\textit{#2}}
  \label{fig:#3}
  \end{figure}}

\newcommand{\FigureFull}[3]{\begin{figure}[H]
  \centering
  \includegraphics[width=1.0\textwidth]{#1}
  \caption{\textit{#2}}
  \label{fig:#3}
  \end{figure}}


\begin{document}

\title{Game Engine Programming \\
  \large Lab 2: The Component Entity System}

\author{Karsten Pedersen\\ Department of Creative Technology}
\maketitle

In this lab we are going to be focusing on the core design of our
engine. This actually means that we wont be touching OpenGL (or DirectX)
just yet.  Instead we want to look at creating an easy and flexible
architecture which will stand the test of time and hopefully ensure that
yourself or anyone using your engine to make an actual game will have
a pleasant experience. Some factors which detract from this include:

\begin{itemize}
\item Large amounts of code required for simple tasks
\item Slow compile times
\item Inflexbile
\item Awkward or unclear API
\item Too easy to make errors
\end{itemize}

As we go through this series of labs, we will cover a number of ways to
avoid these problems. First lets look at a number of other engines and
see how they work.

Lets say a task we want to achieve is to add a simple cube to the scene
at a certain position and rotation. The first example we will use is
Unreal Engine 4:

\begin{Code}

// Create actor at specific location
FActorSpawnParameters si;
GetWorld()->SpawnActor<AActor>(FVector(), FRotator(0.0f, 45.0f, 0.0f), si);

// Create mesh
UStaticMeshComponent* mesh =
  CreateDefaultSubobject<UStaticMeshComponent>(TEXT("Cube Mesh"));

// Assign it to the actor
actor->SetRootComponent(mesh);

\end{Code}

In general, the process is quite straight forward, however some of
the types are a little awkward to remember and quite long to type
such as \Name{FActorSpawnParameters}, \Name{UStaticMeshComponent} and
\Name{CreateDefaultSubobject}. The \Name{U}, \Name{A} and \Name{F}
prefixes are part of the coding standard which can be a little
confusing. The requirement of a \Name{FActorSpawnParameters} reference is
also quite verbose. The \Name{TEXT} MACRO is also easy to forget unless
you come from a C background.

Now there are a number of technical reasons for this design such as
avoiding namespaces, unicode and even to avoid \Name{std::string} because
in the early days, this was still a little bit too young or expensive in
terms of performance. Some of these issues you might want to find other
solutions to in your design.

\Note{
At first glance one of the largest issues you might see with UE4 is
that it is using \Name{raw pointers} for \Name{C++} classes. This is
extremely bad coding practice. However, if you did a little deeper,
you will see that the entire memory management system of UE4 is actually
based on \Name{Garbage Collection} and \Name{Virtual Memory} providing
a good amount of safety at the expense of a little bit of performance
and portability to otehr platforms. Whilst I would not advise using this
for your engine, you might like to read about this more:

\begin{itemize}
\item \url{https://wiki.unrealengine.com/index.php?title=Garbage_Collection\_\%26\_Dynamic\_Memory\_Allocation}
\item \url{http://man7.org/linux/man-pages/man2/mprotect.2.html}
\item \url{https://msdn.microsoft.com/en-us/library/windows/desktop/aa366898(v=vs.85).aspx}
\item \url{http://www.hboehm.info/gc/}
\end{itemize}

}

The second example is Unity. You should see quite a difference here
because unlike UE4 which has evolved as needed by Epic Games to support
their requirements internally (whilst trying hard to avoid it becoming
a sprawling mess!), Unity has been engineered from the ground up to be
an easy to use prosumer engine. This has lead to a much more straight
forward API to those less familiar with 3D software evelopment.

\begin{Code}

// Create GameObject and set location
GameObject go = new GameObject();
go->GetComponent<Transform>().eulerAngles = new Vector3(0, 45.0f, 0);

// Create a mesh
Mesh mesh = Resources.Load<Mesh>("CubeMesh");

// Add a mesh renderer
MeshRenderer mr = go.AddComponent<MeshRenderer>();

// Add a mesh filter and assign our loaded mesh
MeshFilter mf = go.AddComponent<MeshFilter>();
mf.mesh = mesh;

\end{Code}

Some of the most immediate areas of interest is the use of public
variables to set (i.e \Name{eularAngles}). These are likely to be
using the properties feature of C\# rather than \textit{setters} and
\textit{getters} which can sometimes lead to more expensive code being
executed than the developer first assumed. Perhaps the biggest issue to
note is that a few more lines are needed to achieve the same result as in
UE4 due to the addition of a \Name{MeshFilter}. You might at first wonder
why not just assign the \Name{Mesh} directly to the \Name{MeshRenderer}?
The main reason I suspect they have done this is to further separate
out the components for better use within their editor. Writing code to
generate objects in this way is not the most recommended method of doing
things within the world of Unity, instead dragging and dropping within
the editor is preferred. This unfortunately leads to quite a bit more
code required to perform simple tasks.

\

So now it is our turn. Since we don't have the baggage of legacy
requirements, the complex requirement of fun-to-use tools or scripting
languages, we can attempt to make our API as simple to use and as
efficient as possible whilst still retaining a powerful feature set.

\Note{At this point you may want to have a lok at some of the other
popular engines in use today such as \Name{CryEngine}, \Name{Source
Engine} and \Name{Godot}.  This kind of research will not only be good
for your assignment but will also help provide a good showcase of what
works well and what could potentially become a maintenance nightmare
at the end.  Perhaps attempt the same example of adding a cube to the
scene using pure code and try to compare the differences.

}

The following is an example API that you might decide upon. I basically
came up with this by looking around at other engines whilst playing
around in a text editor and a UML diagram tool until I found something
that I wasn't going to get sick of using throughout this unit. So I
suggest you do the same instead of just copying this code.

\begin{Code}

#include <myengine/myengine.h>

#include <memory>

using namespace myengine;

int main()
{
  // Initialize our engine
  std::shared_ptr<Core> core = Core::initialize();

  // Create a single in-game object
  std::shared_ptr<Entity> entity = core->addEntity();

  // Add a very simple component to it
  std::shared_ptr<TestScreen> testScreen = entity->addComponent<TestScreen>();

  // Start the engine's main loop
  core->start();

  return 0;
}

\end{Code}

Ignoring the fact that in this example we have also needed to include
the headers, add a \Name{main} function and started up the engine, you
will see that there are only two lines needed to add an \Name{Entity}
and an example \Name{TestScreen} to it.

\

Smart pointers and RAII provides \Name{C++} with fantastic (and quite
unique) functionality when dealing with memory, however it is a little
bit long winded to type.  We can address this by using \Name{typedefs}
or \Name{MACROS}. The simplest way would be to define in your engine's
headers the following:

\begin{Code}
#include <memory>

#define shared std::shared_ptr
#define weak std::weak_ptr
\end{Code}

Now in your code, it would more look something like:

\begin{Code}
// Create a single in-game object
shared<Entity> entity = core->addEntity();

// Add a very simple component to it
weak<TestScreen> testScreen = entity->addComponent<TestScreen>();
\end{Code}

\Note{
The \Name{MACRO} or \Name{typedef} approach will be great to see
in your \textbf{game} to demonstrate the use of the engine but I
might suggest that it will be easier for me to follow your code
if in your \textbf{engine} you stick to using the full name, i.e
\Name{std::shared\_ptr}. Remember, they should end up completely
interchangable anyway.

}

So now I am going to leave you to it to explore and come up with your own
engine design you are happy with.  I have added a UML diagram describing
the approach that I am going to take in \Ref{early}, yours might end up
being similar but certainly does not have to if you have a good reason
for it.

\FigureFull{images/uml}{The structure of an early game engine}{early}

A small number of points that you might want to think about:

\begin{itemize}

\item Why do I require \Name{core::addEntity} rather than \Name{new Entity}?
\item Why do I avoid static classes like \Name{Application, Environment, Keyboard} and instead have \Name{getters} in the \Name{Component} class?
\item Why do I have multiple \Name{Entity::addComponent} template functions?

\end{itemize}

\end{document}
