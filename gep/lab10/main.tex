\documentclass[10pt]{article}

\usepackage[a4paper, total={6in, 10in}]{geometry}
\usepackage{fancyvrb}
\usepackage{color}
\usepackage{xcolor}
\usepackage{framed}
\usepackage{environ}
\setlength{\parindent}{0pt}

%\usepackage{fancyvrb}
\usepackage{xcolor}
\usepackage{float}
\usepackage{graphicx}
\usepackage{listings}
\usepackage{mdframed}
\usepackage{tcolorbox}

\definecolor{customnavy}{RGB}{0,0,100}

\lstnewenvironment{CodeInner}{%
}{%
}

\newenvironment{CodeOld}{%
\begin{CodeInner}}{%
\end{CodeInner}}

%\newenvironment{Code}
%{\VerbatimEnvironment
%\begin{mdframed}[backgroundcolor=customnavy,fontcolor=white,linewidth=0]
%\begin{Verbatim}
%}{\end{Verbatim}\end{mdframed}}

\newenvironment{Code}
{\VerbatimEnvironment
\begin{tcolorbox}[arc=0pt,outer arc=0pt,before skip=10px, after skip=10px, colframe=white, colback=customnavy, fontupper=\color{white}]
\begin{Verbatim}
}{\end{Verbatim}\end{tcolorbox}}

\newcommand{\Todo}[1]{\begin{center}\colorbox{red!30}{\parbox{0.9\textwidth}{\textbf{\textit{TODO}}\\#1}}\end{center}}
\newcommand{\Note}[1]{\begin{center}\colorbox{gray!30}{\parbox{1.0\textwidth}{\textbf{\textit{Note:}}\\#1}}\end{center}}

\newcommand{\Name}[1]{\textit{#1}}
\newcommand{\Command}[1]{\textbf{\textit{#1}}}

\newcommand{\Ref}[1]{\textbf{Figure \ref{fig:#1}}}
\newcommand{\Sec}[1]{\textbf{Section \ref{sec:#1}}}

\newcommand{\Diagram}[1]{\begin{figure}[H]
  \centering
  \includegraphics[width=0.5\textwidth]{#1}
  \end{figure}}

\newcommand{\Figure}[3]{\begin{figure}[H]
  \centering
  \includegraphics[width=0.5\textwidth]{#1}
  \caption{\textit{#2}}
  \label{fig:#3}
  \end{figure}}

\newcommand{\FigureFull}[3]{\begin{figure}[H]
  \centering
  \includegraphics[width=1.0\textwidth]{#1}
  \caption{\textit{#2}}
  \label{fig:#3}
  \end{figure}}


\begin{document}

\title{Game Engine Programming \\
  \large Lab 10: Multiple Cameras}

\author{Karsten Pedersen\\ Department of Creative Technology}
\maketitle

In this lab we will look into implementing multiple cameras into the engine
in order to open up new functionality such as split screen local multiplayer,
render textures and post processing.

Lets begin by looking at our existing code:

\begin{Code}

while(running)
{
  while (SDL_PollEvent(&event))
  {
    if(event.type == SDL_QUIT)
    {
      running = false;
    }
  }

  getWorld()->tick();
  getWorld()->display();
}

\end{Code}

This allows the game loop to function and each frame we update the world in
\Name{tick} and then draw the world via \Name{display}. We will focus on
the rendering of the world. In your \Name{MeshRenderer} class or alternative
you should have something similar to:

\begin{Code}

std::shared_ptr<Entity> ce = getWorld()->getEntity<Camera>();

shader->setModel(getTransform()->getModelMatrix());
shader->setView(glm::inverse(ce->getTransform()->getModelMatrix());
shader->setProjection(ce->getComponent<Camera>()->getProjection());

\end{Code}

This makes the assumption that there is a single camera in the world and
in order to retrieve it we scan the entities for one with that component
attached. Instead lets store the \Name{Current} camera as a reference
whilst we draw the scene once for each camera. We should change our main loop
code to something similar to:

\begin{Code}

while(running)
{
  while (SDL_PollEvent(&event))
  {
    if(event.type == SDL_QUIT)
    {
      running = false;
    }
  }

  getWorld()->tick();
  std::vector<std::shared_ptr<Entity> > ces;
  getWorld->getEntities<Camera>(ces);

  for(size_t i = 0; i < cameras.size(); i++)
  {
    getWorld()->setCurrentCamera(cameras.at(i)->getComponent<Camera>());
    getWorld()->display();
  }
}

\end{Code}

The key part here is using \Name{setCurrentCamera} for each camera we iterate
through the scene for. This means later on in the \Name{MeshRenderer} we can
retrieve it and use it for future rendering. The following listing demonstrates
this:

\begin{Code}

shader->setView(glm::inverse(
  getWorld()->getCurrentCamera()->getTransform()->getModel()));

shader->setProjection(getWorld()->getCurrentCamera()->getProjectionMatrix());

\end{Code}

\Note{
  With this in place; see if you can render a few instances of your
  scene from different locations and either toggle between them or draw
  them at different parts of the screen. The function \Name{glViewport}
  might be useful here.
}

\Note{
  Have a look into how Unreal Engine or Unity deals with cameras and
  render targets. Have a read up on using OpenGL to create Frame Buffer
  Objects and this is what we will be looking at attaching to each camera next.
}

\end{document}
