\documentclass[10pt]{article}

\usepackage[a4paper, total={6in, 10in}]{geometry}
\usepackage{fancyvrb}
\usepackage{color}
\usepackage{xcolor}
\usepackage{framed}
\usepackage{environ}
\setlength{\parindent}{0pt}

\newenvironment{Listing}
{\VerbatimEnvironment
\begin{Verbatim}[frame=lines]}
{\end{Verbatim}}

%\usepackage{fancyvrb}
\usepackage{xcolor}
\usepackage{float}
\usepackage{graphicx}
\usepackage{listings}
\usepackage{mdframed}
\usepackage{tcolorbox}

\definecolor{customnavy}{RGB}{0,0,100}

\lstnewenvironment{CodeInner}{%
}{%
}

\newenvironment{CodeOld}{%
\begin{CodeInner}}{%
\end{CodeInner}}

%\newenvironment{Code}
%{\VerbatimEnvironment
%\begin{mdframed}[backgroundcolor=customnavy,fontcolor=white,linewidth=0]
%\begin{Verbatim}
%}{\end{Verbatim}\end{mdframed}}

\newenvironment{Code}
{\VerbatimEnvironment
\begin{tcolorbox}[arc=0pt,outer arc=0pt,before skip=10px, after skip=10px, colframe=white, colback=customnavy, fontupper=\color{white}]
\begin{Verbatim}
}{\end{Verbatim}\end{tcolorbox}}

\newcommand{\Todo}[1]{\begin{center}\colorbox{red!30}{\parbox{0.9\textwidth}{\textbf{\textit{TODO}}\\#1}}\end{center}}
\newcommand{\Note}[1]{\begin{center}\colorbox{gray!30}{\parbox{1.0\textwidth}{\textbf{\textit{Note:}}\\#1}}\end{center}}

\newcommand{\Name}[1]{\textit{#1}}
\newcommand{\Command}[1]{\textbf{\textit{#1}}}

\newcommand{\Ref}[1]{\textbf{Figure \ref{fig:#1}}}
\newcommand{\Sec}[1]{\textbf{Section \ref{sec:#1}}}

\newcommand{\Diagram}[1]{\begin{figure}[H]
  \centering
  \includegraphics[width=0.5\textwidth]{#1}
  \end{figure}}

\newcommand{\Figure}[3]{\begin{figure}[H]
  \centering
  \includegraphics[width=0.5\textwidth]{#1}
  \caption{\textit{#2}}
  \label{fig:#3}
  \end{figure}}

\newcommand{\FigureFull}[3]{\begin{figure}[H]
  \centering
  \includegraphics[width=1.0\textwidth]{#1}
  \caption{\textit{#2}}
  \label{fig:#3}
  \end{figure}}


%{\colorbox{blue!30}{\BODY}}

%\newcommand{\Note}[1]{\par\colorbox{blue!30}{#1}}

\begin{document}

\title{Game Engine Programming \\
  \large Lab 1: An Introduction to CMake}

%\subject{Game Engine Programming}
%\subtitle{Lab 1 An Introduction to CMake}
\author{Karsten Pedersen\\ Department of Creative Technology}
\maketitle

In this unit we aim to build a game engine which will potentially generate
a number of libraries and will likely have a number of dependencies
itself on 3rd party libraries. We will also be looking at alternative
C++ compilers to the one provided by \textit{Microsoft Visual Studio}
(\textbf{\textit{cl}}). For this reason we will be using \textit{CMake}
to manage our project for us.

\

The first thing we will need to do is ensure that it is installed and
added to our \textbf{\textit{PATH}} environment variable.  To do this,
open up a command prompt or terminal emulator and enter:

\begin{Code}
  > cmake --version
\end{Code}

If \textit{CMake} is correctly installed, you should see output similar to:

\begin{Code}
  cmake version 2.8.12.2
\end{Code}

Hopefully with that in place we can start creating our initial
project. The first step is to create a new directory called
\Command{myengine} (or anything you want to call your game engine). Try
to avoid network drives such as your \textbf{\textit{H:\textbackslash}}
because it is fairly slow to build software. Once you have created
this directory, we will populate it with a general framework of the
project. \Ref{simple} shows the beginning of a standard C/C++ project
layout.

\Figure{images/simple}{Diagram showing the intended filesystem structure}{simple}

To get started the \textbf{\textit{main.cpp}} should be populated with a small \textit{Hello World} program as shown in the listing below.

\begin{Code}
  #include <iostream>

  int main()
  {
    std::cout << "Hello World" << std::endl;

    return 0;
  }
\end{Code}

Next the file called \textbf{\textit{CMakeLists.txt}} is what instructs
\textit{CMake} how to build your program from the source.  It can describe
every aspect about building your project from a single file or multiple
depending on how complex your build system gets. For example it could
be instructed to perform tasks such as running test suites, generating
documentation, baking lighting, compiling to multiple platforms, building
installers, etc. However, for now lets start with the bare minimum.

\begin{Code}
  cmake_minimum_required(VERSION 2.6)
  project(MYENGINE)

  add_executable(game
    src/game/main.cpp
  )
\end{Code}

The \textbf{\textit{cmake\_minimum\_required}} statement tells
\textit{CMake} that it only uses older features of the build tool. This
allows for great backwards compatibility with less updated platforms. As
we require more modern features, we can bump up the version to unlock them
at the expense of supporting old operating systems such as \textit{Solaris
10} or \textit{Microsoft Windows XP}.

The \textbf{\textit{project}} statement tells \textit{CMake} what to
name the entire project (not just individual libraries or executables
that it generates). This is useful when creating scripts later on. We
use all capitals here because otherwise variables are named things like
\textbf{\textit{myengine\_VERSION}} which looks a bit horrible.

Finally the \textbf{\textit{add\_executable}} command instructs
\textit{CMake} to create an output executable (\textit{*.exe}
on \textit{Windows}) with the given name and compiled from the
specified source files. We could generate a library (\textit{*.lib}
for \textit{Microsoft Visual C++}, \textit{*.a} for \textit{GNU C/C++ Compiler})
by using \textbf{\textit{add\_library}} instead. You will encounter this
later on.

\

Once this file is in place, now we build the project. First we create
a new directory called \textbf{\textit{build}} in the root of the
project.  This is where \textit{CMake} puts \textbf{all} of the generated
files. This is an especially nice feature because it doesn't spam
your project with clutter.  Then you can easily delete the folder to
clean the project or simply add it to the \textbf{\textit{.gitignore}}
file to prevent accidentally committing it into your \textit{Git}
repository. We then use \textit{CMake} to generate the project and
finally use \textit{CMake} again to build it. The following listing
shows this process.

\begin{Code}
  > mkdir build     # Create the directory
  > cd build        # Go into the newly created directory

  > cmake ..        # Generate project in this directory but use the
                    # CMakeLists.txt file from one level up (..)

  > cmake --build . # Build from the files in current directory (.)
  > Debug\game.exe  # Run the created executable
\end{Code}

From now on, to trigger rebuilds, simply use the
\textbf{\textit{cmake --build .}} unless you change machines or delete
the \textbf{\textit{build}} folder.  Remember however to make sure you
are in the correct directory. I personally move back into the root of
the project and compile from there. For example:

\begin{Code}
  > cd ..
  > cmake --build build
\end{Code}

This has the small benefit that you can still use \textit{Git} easily
from the root of the project and also any relative paths that a simple
naive program may try to open will work correctly.

\

Finally, the whole point of \textit{CMake} is to use existing
tools. If you want to use \textit{Microsoft Visual Studio}, you
can do so by simply double clicking on the \textbf{\textit{*.sln}}
file within the created \textbf{\textit{build}} directory. You
can then pretty much use it as normal. Just remember to use the
\textbf{\textit{CMakeLists.txt}} file instead to add and remove
\textbf{\textit{*.h}} and \textbf{\textit{*.cpp}} files.

\Note{
Spend this time at the end of the lab setting up a version control repo
for your project (i.e \textit{Git}, \textit{Subversion}). Your project
will likely grow quite large and source control will help you manage it.
Simply check in everything apart from the \Command{build} folder. This
could be added to your \Command{.gitignore} file. With a bit of practice,
you will start to see that \Name{Git} and \Name{CMake} work very well
together and why they are such common tools to use within the industry.

\

Once you are happy with using \textit{CMake}. Try adding a new subfolder
to the \textbf{\textit{src}} directory called \textbf{\textit{myengine}}
and start adding some initial files in there that you think you might
need. Try to compile them into a library using \Command{add\_library}
in your \Command{CMakeLists.txt}. At this point, your project layout
will start to begin looking similar to \Ref{additional}.

}

\FigureFull{images/additional}{The structure of a slightly more developed project}{additional}

\end{document}
